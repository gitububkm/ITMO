\documentclass[12pt]{article}
\usepackage[T2A]{fontenc}
\usepackage[utf8]{inputenc}
\usepackage[russian]{babel}
\usepackage{amsmath,amssymb}
\usepackage{geometry}
\geometry{a4paper,margin=2cm}
\setlength{\parindent}{0pt}
\setlength{\parskip}{0.6em}

\newcommand{\Z}{\mathbb{Z}}
\let\NOD\relax
\newcommand{\NOD}{\operatorname{НОД}}

\begin{document}

\begin{center}
    {\Large Домашнее задание №1}\\[0.4em]
    \textit{Бардышев Артём Антонович 408221}
\end{center}

\textbf{Задание 1.} Найдите остаток от деления
\[
6^{93}\cdot 8^{90} + 50^{12}\cdot 90^{10} \quad \text{на } 47.
\]

\underline{Решение.}
\begin{enumerate}
    \item \textbf{Понижение степеней первого множителя.} Сначала заметим, что $6\cdot 8 = 48 \equiv 1 \pmod{47}$. Это означает, что числа $6$ и $8$ взаимно обратны по модулю $47$, то есть $8 \equiv 6^{-1} \pmod{47}$. Значит, из произведения $6^{93}\cdot 8^{90}$ можно ``погасить'' большую часть степеней:
    \[
    6^{93}\cdot 8^{90} \equiv 6^{93}\cdot (6^{-1})^{90} = 6^{93-90} = 6^{3}.
    \]
    Осталось сократить $6^{3}$ по модулю $47$: $6^{3}=216$, а $216 = 47\cdot 4 + 28$, значит $6^{93}\cdot 8^{90} \equiv 28 \pmod{47}$.

    \item \textbf{Приведение второго множителя.} Числа $50$ и $90$ тоже удобно заменить остатками: $50 \equiv 3 \pmod{47}$, $90 \equiv -4 \pmod{47}$. Тогда
    \[
    50^{12}\cdot 90^{10} \equiv 3^{12}\cdot (-4)^{10}.
    \]
    Степень $12$ у тройки можно считать последовательно: $3^2=9$, $3^3=27$, $3^4=81\equiv34$, далее $3^5\equiv6$, $3^6\equiv18$, $3^7\equiv8$, $3^8\equiv24$, $3^9\equiv25$, $3^{10}\equiv31$, $3^{11}\equiv46$, $3^{12}\equiv12$. Для степени $10$ удобно заметить, что $(-4)^{10}=4^{10}$, а $4^2=16$, $4^3\equiv64\equiv17$, $4^4\equiv68\equiv21$, $4^5\equiv84\equiv-10$, $4^{10}\equiv(-10)^2\equiv100\equiv6$. Значит, вторая часть равна $12\cdot6\equiv72\equiv25 \pmod{47}$.

    \item \textbf{Сборка результата.} Складываем остатки частей: $28+25=53$, а $53=47+6$, следовательно итоговый остаток $6$.
\end{enumerate}

Ответ: остаток равен $6$.

\vspace{0.6em}
\textbf{Задание 2.} Используя признак Паскаля, проверьте, делится ли число $2131BB9BCA_{13}$ на $18$.

\underline{Решение.}
\begin{enumerate}
    \item \textbf{Подготовка коэффициентов.} Признак Паскаля основан на том, что $13 \equiv -5 \pmod{18}$. Значит $13^2 \equiv 25 \equiv 7$, а $13^3 \equiv -35 \equiv 1 \pmod{18}$. Поэтому каждую третью степень можно заменить единицей, и удобно группировать цифры по три.

    \item \textbf{Разбиение числа на триплеты.} Записываем число блоками справа налево:
    \[
    213\,|\,1BB\,|\,9BC\,|\,A.
    \]
    Первому блоку соответствует коэффициент $1$, второму --- $-5$, третьему --- $7$, четвёртому снова $-5$ (коэффициенты повторяются с периодом $1,-5,7$).

    \item \textbf{Перевод блоков и вычисление суммы.} 
    \begin{align*}
        (213)_{13} &= 2\cdot 13^2 + 1\cdot 13 + 3 = 2\cdot169 + 13 + 3 = 354,\\
        (1BB)_{13} &= 1\cdot 13^2 + 11\cdot 13 + 11 = 169 + 143 + 11 = 323,\\
        (9BC)_{13} &= 9\cdot 13^2 + 11\cdot 13 + 12 = 9\cdot169 + 143 + 12 = 1666,\\
        A_{13} &= 10.
    \end{align*}
    Объединяем с весами:
    \[
    S = 354\cdot 1 - 323\cdot 5 + 1666\cdot 7 - 10\cdot 5 = 354 - 1615 + 11662 - 50 = 10351.
    \]
    Теперь $10351 \equiv 17 \pmod{18}$.

    \item \textbf{Вывод.} Остаток $17$ не равен нулю, значит исходное число не делится на $18$.
\end{enumerate}

\vspace{0.6em}
\textbf{Задание 3.} Существует ли целое $n$, при котором дробь
\[
\frac{50n^2+70n+4}{100n^2+150n+9}
\]
сократима?

\underline{Решение.}
Обозначим числитель и знаменатель через
\[
A(n)=50n^2+70n+4,\qquad B(n)=100n^2+150n+9,
\]
и рассмотрим их НОД: $d=(A(n),B(n))$. Будем понижать пару, как в алгоритме Евклида.
\begin{enumerate}
    \item \textbf{Первый шаг.} Вычитаем удвоенный числитель из знаменателя:
    \[
    (A,B) = (A,\;B-2A) = \big(A,\;10n+1\big).
    \]
    Теперь вместо громоздких квадратов имеем линейный многочлен $10n+1$.

    \item \textbf{Второй шаг.} Снова применяем тот же приём:
    \[
    (A,\;10n+1) = \big(A - (5n+7)(10n+1),\;10n+1\big) = (-25,\;10n+1).
    \]
    То есть НОД наших многочленов совпадает с НОД чисел $25$ и $10n+1$.

    \item \textbf{Финал.} Значит $d = (25,\;10n+1)$. Но $10n+1 \equiv 1 \pmod 5$ при любом целом $n$, поэтому общий делитель не может содержать множитель $5$. Следовательно, $d=1$, и исходная дробь несократима.
\end{enumerate}

\vspace{0.6em}
\textbf{Задание 4.} Докажите для нечётных целых $a,b,c$ тождество
\[
\NOD\!\left(\frac{a+b}{2},\frac{a+c}{2},\frac{b+c}{2}\right) = \NOD(a,b,c).
\]

\underline{Доказательство.}
\begin{enumerate}
    \item \textbf{Корректность выражений.} Поскольку $a,b,c$ нечётные, их попарные суммы чётны, значит каждое из чисел $\frac{a+b}{2}$, $\frac{a+c}{2}$, $\frac{b+c}{2}$ действительно целое.

    \item \textbf{Покажем, что левый НОД делит правый.} Обозначим $g = \NOD\!\left(\frac{a+b}{2},\frac{a+c}{2},\frac{b+c}{2}\right)$. Выразим исходные числа через половины сумм:
    \[
    a = \frac{a+b}{2} + \frac{a+c}{2} - \frac{b+c}{2},\qquad
    b = \frac{a+b}{2} + \frac{b+c}{2} - \frac{a+c}{2},
    \]
    и т.д. Линейные комбинации показывают, что $g$ делит каждое из $a,b,c$. Значит $g \mid \NOD(a,b,c)$.

    \item \textbf{Покажем обратное деление.} Пусть теперь $d = \NOD(a,b,c)$. Тогда $d$ делит каждое из $a,b,c$, а значит делит их суммы. Поскольку суммы чётные, деление на $2$ не нарушает целостности, и $d$ делит каждую половину суммы. Значит $d \mid g$.

    \item \textbf{Заключение.} Мы получили взаимное деление $g \mid d$ и $d \mid g$, откуда $g=d$, что и требовалось.
\end{enumerate}

\end{document}

